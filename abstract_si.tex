Tvorjenje naravnega jezika je področje obdelave naravnega jezika, ki se ukvarja z tvorjenjem besedil, za katera bi bralci lahko utemeljeno sklepali, da so človeškega izvora. Cilj diplomsko dela je ugotoviti do kolikšne mere lahko kompleksna pravila naravnega jezika posnemamo s statističnimi sistemi, natančneje s skritimi markovskimi modeli. Diplomsko delo predstavlja potrebno teoretično podlago za obstoj skritih markovskih modelov, opisuje uporabo le-teh za tvorjenje besedil in predstavi, kako vse to uporabimo v praksi. Opravi se pregled obstoječih orodij za delo s skriti markovskimi modeli, medsebojno primerjavo orodij in pregled njihove primernosti za uporabo pri tvorjenju besedil. Opiše se postopek implementacije knjižnice za delo s skritimi markovskimi modeli in težave, ki pri tem nastanejo. Dve izmed pregledanih orodij in implementirano knjižnico se uporabi za tvorjenje besedil na podlagi korpusa slovenskega pisnega jezika. Orodja se primerjajo med seboj na podlagi tvorjenih besedil. Opravi se tudi primerjava tvorjenih besedil s korpusom.
