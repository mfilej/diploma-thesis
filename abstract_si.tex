Področje NLG se ukvarja s tvorjenjem naravno zvenečih besedil.
Cilj diplomskega dela je ugotoviti, do kolikšne mere lahko kompleksna pravila tvorjenja naravnega jezika posnemamo s statističnimi sistemi, natančneje s skritimi markovskimi modeli. 
Delo predstavi potrebno teoretično podlago za obstoj skritih markovskih modelov in opiše njihovo uporabo pri tvorjenju besedil.
V okviru diplomskega dela je opravljen tudi pregled obstoječih orodij za delo s skriti markovskimi modeli, medsebojna primerjava orodij in pregled njihove primernosti za uporabo pri tvorjenju besedil.
Opisan je postopek implementacije knjižnice za delo s skritimi markovskimi modeli v programskem jeziku Elixir.
Dve izmed pregledanih orodij in implementirana knjižnica so uporabljeni za tvorjenje besedil na podlagi korpusa slovenskega pisnega jezika.
Izbere se kriterij za primerjavo tvorjenih besedil, ki se uporabi za primerjavo modelov, kot tudi za primerjavo tvorjenih besedil s korpusom.
