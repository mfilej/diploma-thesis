Tvorjenje naravnega jezika je del področja NLP (obdelava naravnega jezika) in se ukvarja s tvorjenjem besedil, za katera bralci lahko utemeljeno sklepajo, da so človeškega izvora.
Naše diplomsko delo se ukvarja z vprašanjem, do kolikšne mere lahko kompleksna pravila tvorjenja naravnega jezika posnemamo s statističnimi sistemi, natančneje s skritimi markovskimi modeli. 
Delo predstavi potrebno teoretično podlago za obstoj skritih markovskih modelov, opiše uporabo le-teh za tvorjenje besedil in ponazori njihovo uporabo v praksi.
V okviru diplomskega dela je opravljen tudi pregled obstoječih orodij za delo s skriti markovskimi modeli, medsebojna primerjava orodij in pregled njihove primernosti za uporabo pri tvorjenju besedil.
Opisan je postopek implementacije knjižnice za delo s skritimi markovskimi modeli in
težave, ki pri tem nastanejo.
Dve izmed pregledanih orodij in implementirana knjižnica so uporabljeni za tvorjenje besedil na podlagi korpusa slovenskega pisnega jezika.
Orodja so med seboj primerjana na podlagi tvorjenih besedil.
V diplomskem delu je opravljena tudi primerjava tvorjenih besedil s korpusom.
