%%%%%%%%%%%%%%%%%%%%%%%%%%%%%%%%%%%%%%%%
% LaTeX template based on
% http://www.fri.uni-lj.si/file/189169/vzorec160416.zip

\documentclass[a4paper, 12pt]{book}
\RequirePackage{pdf14}

\usepackage[a-2b]{pdfx}
\usepackage[T1]{fontenc}
\usepackage[utf8]{inputenc}
\usepackage{type1ec}
\usepackage{fancyhdr}               % poskrbi, na primer, za glave strani
\usepackage[pdftex]{graphicx}       % omogoča vlaganje slik različnih formatov
\usepackage[slovene,english]{babel} % naloži, med drugim, slovenske delilne vzorce
\usepackage{cmap}
\usepackage{amssymb}                % dodatni simboli
\usepackage{amsmath}                % eqref, npr.
\usepackage{amsthm}                 % hipoteze
\usepackage{perpage}
\usepackage{caption}
\usepackage{subcaption}
\usepackage{enumitem}
\usepackage{multirow}
\usepackage{wrapfig,booktabs}
\usepackage{xcolor} % \wip command
\usepackage[chapter]{algorithm}
\usepackage[noend]{algpseudocode}
\usepackage[scale=2]{ccicons}                % CC license symbols
\usepackage{url}
\usepackage{hyperref}
\usepackage[
  style=ieee, urldate=comp, sorting=nty, dateabbrev=false, sortcites
]{biblatex}

%\usepackage{hyperxmp}
\hypersetup{pdftex, colorlinks=true,
  citecolor=black, filecolor=black, 
  linkcolor=black, urlcolor=black,
  pagebackref=false, 
  pdfproducer={LaTeX}, pdfcreator={LaTeX}, hidelinks}

%%%%%%%%%%%%%%%%%%%%%%%%%%%%%%%%%%%%%%%%
% Document info
%%%%%%%%%%%%%%%%%%%%%%%%%%%%%%%%%%%%%%%%
\newcommand{\ttitle}{Tvorjenje besedil z uporabo skritega markovskega modela}
\newcommand{\ttitleEn}{Text Generation using Hidden Markov Model}
\newcommand{\tsubject}{\ttitle}
\newcommand{\tsubjectEn}{\ttitleEn}
\newcommand{\tauthor}{Miha Filej}
\newcommand{\tauthoremail}{miha@filej.net}
\newcommand{\tkeywords}{tvorjenje naravnega jezika, skriti markovski modeli,
algoritem Baum-Welch, algoritem Forward-Backward, algoritem EM,
Elixir, Erlang/OTP}
\newcommand{\tkeywordsEn}{natural language generation, hidden markov models,
Baum-Welch algorithm, Forward-Backward algorithm, expectation–maximization
algorithm, Elixir, Erlang/OTP}

%%%%%%%%%%%%%%%%%%%%%%%%%%%%%%%%%%%%%%%%
% Hyperref setup
%%%%%%%%%%%%%%%%%%%%%%%%%%%%%%%%%%%%%%%%
\usepackage{hyperref}
\hypersetup{pdftitle={\ttitle}}
\hypersetup{pdfsubject=\ttitleEn}
\hypersetup{pdfauthor={\tauthor, \tauthoremail}}
\hypersetup{pdfkeywords=\tkeywordsEn}

%%%%%%%%%%%%%%%%%%%%%%%%%%%%%%%%%%%%%%%%
% Page setup
%%%%%%%%%%%%%%%%%%%%%%%%%%%%%%%%%%%%%%%%
\addtolength{\marginparwidth}{-20pt} % robovi za tisk
\addtolength{\oddsidemargin}{40pt}
\addtolength{\evensidemargin}{-40pt}

\renewcommand{\baselinestretch}{1.3} % ustrezen razmik med vrsticami
\setlength{\headheight}{15pt}        % potreben prostor na vrhu
\renewcommand{\chaptermark}[1]%
{\markboth{\MakeUppercase{\thechapter.\ #1}}{}} \renewcommand{\sectionmark}[1]%
{\markright{\MakeUppercase{\thesection.\ #1}}} \renewcommand{\headrulewidth}{0.5pt} \renewcommand{\footrulewidth}{0pt}
\fancyhf{}
\fancyhead[LE,RO]{\sl \thepage} \fancyhead[LO]{\sl \rightmark} \fancyhead[RE]{\sl \leftmark}

\newcommand{\BibTeX}{{\sc Bib}\TeX}

\MakePerPage{footnote} % restart footnote numbering on every page

\raggedbottom

%%%%%%%%%%%%%%%%%%%%%%%%%%%%%%%%%%%%%%%%
% Headings
%%%%%%%%%%%%%%%%%%%%%%%%%%%%%%%%%%%%%%%%
\newcommand{\autfont}{\Large}
\newcommand{\titfont}{\LARGE\bf}
\newcommand{\clearemptydoublepage}{\newpage{\pagestyle{empty}\cleardoublepage}}
\setcounter{tocdepth}{1}	      % globina kazala

%%%%%%%%%%%%%%%%%%%%%%%%%%%%%%%%%%%%%%%%
% Theorems
%%%%%%%%%%%%%%%%%%%%%%%%%%%%%%%%%%%%%%%%
\newtheorem{izrek}{Izrek}[chapter]
\newtheorem{trditev}{Trditev}[izrek]
\newenvironment{dokaz}{\emph{Dokaz.}\ }{\hspace{\fill}{$\Box$}}
\newtheorem{hypothesis}{Hipoteza}

%%%%%%%%%%%%%%%%%%%%%%%%%%%%%%%%%%%%%%%%
% Algorithms
%%%%%%%%%%%%%%%%%%%%%%%%%%%%%%%%%%%%%%%%

\makeatletter
\renewcommand{\ALG@name}{Algoritem}
\renewcommand{\listalgorithmname}{Seznam algoritmov}
\makeatother

%%%%%%%%%%%%%%%%%%%%%%%%%%%%%%%%%%%%%%%%
% Misc 
%%%%%%%%%%%%%%%%%%%%%%%%%%%%%%%%%%%%%%%%

\newcommand{\angl}[1][Prevod]{\footnote{angl. \textit{#1}}}
\def\tightlist{} % pandoc compatibility
\newcommand\wip[1]{\textcolor{red}{\small{\texttt{TODO:}} #1}}

\newcommand{\given}{\medspace\lvert\medspace}

\newcommand{\obsseq}[3]{O_{#1} O_{#2} \cdots O_{#3}}

\interfootnotelinepenalty=10000

%%%%%%%%%%%%%%%%%%%%%%%%%%%%%%%%%%%%%%%%%%%%%%%%%%%%%%%%%%%%%%%%%%%%%%%%%%%%%%%
%% PDF-A
%%%%%%%%%%%%%%%%%%%%%%%%%%%%%%%%%%%%%%%%%%%%%%%%%%%%%%%%%%%%%%%%%%%%%%%%%%%%%%%

%%%%%%%%%%%%%%%%%%%%%%%%%%%%%%%%%%%%%%%%
% Define medatata
%%%%%%%%%%%%%%%%%%%%%%%%%%%%%%%%%%%%%%%%
\def\Title{\ttitle}
\def\Author{\tauthor, \tauthoremail}
\def\Subject{\ttitleEn}
\def\Keywords{\tkeywordsEn}

%%%%%%%%%%%%%%%%%%%%%%%%%%%%%%%%%%%%%%%%
% \convertDate converts D:20080419103507+02'00' to 2008-04-19T10:35:07+02:00
%%%%%%%%%%%%%%%%%%%%%%%%%%%%%%%%%%%%%%%%
\def\convertDate{%
    \getYear
}

{\catcode`\D=12
 \gdef\getYear D:#1#2#3#4{\edef\xYear{#1#2#3#4}\getMonth}
}
\def\getMonth#1#2{\edef\xMonth{#1#2}\getDay}
\def\getDay#1#2{\edef\xDay{#1#2}\getHour}
\def\getHour#1#2{\edef\xHour{#1#2}\getMin}
\def\getMin#1#2{\edef\xMin{#1#2}\getSec}
\def\getSec#1#2{\edef\xSec{#1#2}\getTZh}
\def\getTZh +#1#2{\edef\xTZh{#1#2}\getTZm}
\def\getTZm '#1#2'{%
    \edef\xTZm{#1#2}%
    \edef\convDate{\xYear-\xMonth-\xDay T\xHour:\xMin:\xSec+\xTZh:\xTZm}%
}

\expandafter\convertDate\pdfcreationdate 

%%%%%%%%%%%%%%%%%%%%%%%%%%%%%%%%%%%%%%%%
% Get pdftex version string
%%%%%%%%%%%%%%%%%%%%%%%%%%%%%%%%%%%%%%%% 
\newcount\countA
\countA=\pdftexversion
\advance \countA by -100
\def\pdftexVersionStr{pdfTeX-1.\the\countA.\pdftexrevision}

%%%%%%%%%%%%%%%%%%%%%%%%%%%%%%%%%%%%%%%%
% XMP data
%%%%%%%%%%%%%%%%%%%%%%%%%%%%%%%%%%%%%%%%  
\usepackage{xmpincl}
%\includexmp{pdfa-1b}

%%%%%%%%%%%%%%%%%%%%%%%%%%%%%%%%%%%%%%%%
% PDF info
%%%%%%%%%%%%%%%%%%%%%%%%%%%%%%%%%%%%%%%%  
\pdfinfo{%
    /Title    (\ttitle)
    /Author   (\tauthor, \tauthoremail)
    /Subject  (\ttitleEn)
    /Keywords (\tkeywordsEn)
    /ModDate  (\pdfcreationdate)
    /Trapped  /False
}

%%%%%%%%%%%%%%%%%%%%%%%%%%%%%%%%%%%%%%%%
% Bibliography
%%%%%%%%%%%%%%%%%%%%%%%%%%%%%%%%%%%%%%%%  

\addbibresource{bibliography.bib}

% Problem: Mendeley Desktop exports tildes in URLs as `$\sim$`.
% Workaround: define a biblatex source map.
% Source: http://tex.stackexchange.com/a/275538.
\DeclareSourcemap{
  \maps{
    \map{
      \step[fieldsource=url,
            match=\regexp{\$\\sim\$},
            replace=\regexp{\~}]
    }
  }
}

\DeclareFieldFormat{url}{\bibstring{url}\addcolon\space\url{#1}}


%%%%%%%%%%%%%%%%%%%%%%%%%%%%%%%%%%%%%%%%%%%%%%%%%%%%%%%%%%%%%%%%%%%%%%%%%%%%%%%
%%%%%%%%%%%%%%%%%%%%%%%%%%%%%%%%%%%%%%%%%%%%%%%%%%%%%%%%%%%%%%%%%%%%%%%%%%%%%%%

\begin{document}
\selectlanguage{slovene}
\frontmatter
\setcounter{page}{1}
\renewcommand{\thepage}{} % Supposedly fixes numbering issues

%%%%%%%%%%%%%%%%%%%%%%%%%%%%%%%%%%%%%%%%
% Title page
 \thispagestyle{empty}%
   \begin{center}
    {\large\sc Univerza v Ljubljani\\%
      Fakulteta za računalništvo in informatiko}%
    \vskip 10em%
    {\autfont \tauthor\par}%
    {\titfont \ttitle \par}%
    {\vskip 2em \textsc{DIPLOMSKO DELO\\[2mm]
    UNIVERZITETNI ŠTUDIJ RAČUNALNIŠTVA IN INFORMATIKE}\par}%
    \vfill\null%
    {\large \textsc{Mentor}: dr. Andrej Brodnik\par}%
    {\vskip 2em \large Ljubljana, 2016 \par}%
\end{center}
% Empty page
\clearemptydoublepage

%%%%%%%%%%%%%%%%%%%%%%%%%%%%%%%%%%%%%%%%
% Copyright page
\thispagestyle{empty}
\vspace*{8cm}

\begin{center}

Uporaba diplomskega dela je dovoljena pod licenco \emph{Attribution 4.0 International} \texttt{(CC BY 4.0)}. Besedilo licence je dostopno na naslovu 
\href{http://creativecommons.org/licenses/by/4.0/}{http://creativecommons.org/licenses/by/4.0/}.

\bigskip

{\ccby}

\hfill

Izvorna koda diplomskega dela, njeni rezultati in v ta namen razvita programska oprema je ponujena pod licenco \emph{MIT}. Besedilo licence je na voljo v datoteki \texttt{LICENCE}, ki pripada izvorni kodi.

\bigskip

Digitalna oblika diplomskega dela je dostopna na naslovu \url{http://miha.filej.net/diploma-thesis}.

\mbox{}\vfill
\emph{Besedilo je oblikovano z urejevalnikom besedil \LaTeX.}
\end{center}
% Empty page
\clearemptydoublepage

%%%%%%%%%%%%%%%%%%%%%%%%%%%%%%%%%%%%%%%%
% Page 3
\thispagestyle{empty}
\vspace*{4cm}

\noindent
Fakulteta za računalništvo in informatiko izdaja naslednjo nalogo: \\
\emph{\ttitle}
\medskip
\begin{tabbing}
\hspace{32mm}\= \hspace{6cm} \= \kill




Tematika naloge:
\end{tabbing}
Tvorjenje besedil je podpodročje obdelave naravnega jezika. Precej pozornosti
na področju se posveča tvorjenju celotnega besedila. Pri predmetu Digitalna
forenzika študenti raziskujejo kriminalne primere, vendar je zelo zaželjeno,
da je vsak primer različen. V diplomski nalogi preučite možnost samodejnega
tvorjenja besedil pri omenjenem predmetu s pomočjo skritega markovskega
modela. Preverite, katera orodja oziroma okolja obstajajo, in jih ovrednotite
za potrebe svoje naloge.
\vspace{15mm}






\vspace{2cm}

% Empty page
\clearemptydoublepage

%%%%%%%%%%%%%%%%%%%%%%%%%%%%%%%%%%%%%%%%
% Authorship declaration
\vspace*{1cm}
\begin{center}
{\Large \textbf{\sc Izjava o avtorstvu diplomskega dela}}
\end{center}

\vspace{1cm}
\noindent Spodaj podpisani \tauthor{} sem avtor diplomskega dela z naslovom:

\vspace{0.5cm}
\emph{\ttitle}\hspace{5mm}(angl. \emph{\ttitleEn})

\vspace{1.5cm}
\noindent S svojim podpisom zagotavljam, da:
\begin{itemize}
  \item sem diplomsko delo izdelal samostojno pod mentorstvom dr.\ Andreja Brodnika,
  \item so elektronska oblika diplomskega dela, naslov (slov., angl.), povzetek (slov., angl.) ter ključne besede (slov., angl.) identični s tiskano obliko diplomskega dela,
  \item soglašam z javno objavo elektronske oblike diplomskega dela na svetovnem spletu preko univerzitetnega spletnega arhiva.	
\end{itemize}

\vspace{1cm}
\noindent V Ljubljani, dne 7. septembra 2016 \hfill Podpis avtorja:

% Empty page
\clearemptydoublepage


%%%%%%%%%%%%%%%%%%%%%%%%%%%%%%%%%%%%%%%%
% Index page
\pagestyle{empty}
\def\thepage{} % Supposedly fixes numbering issues
\tableofcontents{}


% Empty page
\clearemptydoublepage

%%%%%%%%%%%%%%%%%%%%%%%%%%%%%%%%%%%%%%%%
% Acronyms

\chapter*{Seznam uporabljenih kratic}

\hspace*{-1.5cm}
\begin{tabular}{l|l|l}
  {\bf kratica} & {\bf angleško} & {\bf slovensko} \\
  \hline
  {\bf BSD}  & Berkeley Software Distribution & Berkley distribucija programske opreme \\
  {\bf EM}   & Expectation–Maximization & maksimizacija pričakovane vrednosti \\
  {\bf GPL}  & GNU General Public License & splošno dovoljenje GNU \\
  {\bf HMM}  & Hidden Markov Model & skriti markovski model (SMM) \\
  {\bf LGPL} & GNU Lesser General Public License & splošno večje dovoljenje GNU \\
  {\bf NLG}  & Natural Language Generation & tvorjenje naravnega jezika \\
  {\bf NLP}  & Natural Language Processing & procesiranje naravnega jezika \\
  {\bf OCR}  & Optical Character Recognition & optično prepoznavanje znakov \\
  {\bf OTP}  & Open Telecom Platform & odprta platforma za telekome \\
  {\bf TEI}  & Text Encoding Initiative & iniciativa za kodiranje besedila \\
\end{tabular}



% Empty page
\clearemptydoublepage

%%%%%%%%%%%%%%%%%%%%%%%%%%%%%%%%%%%%%%%%
% Summary page
\addcontentsline{toc}{chapter}{Povzetek}
\chapter*{Povzetek}

\noindent\textbf{Naslov:} \ttitle
\bigskip

\noindentTvorjenje naravnega jezika je del področja NLP (obdelava naravnega jezika) in se ukvarja s tvorjenjem besedil, za katera bralci lahko utemeljeno sklepajo, da so človeškega izvora.
Naše diplomsko delo se ukvarja z vprašanjem, do kolikšne mere lahko kompleksna pravila tvorjenja naravnega jezika posnemamo s statističnimi sistemi, natančneje s skritimi markovskimi modeli. 
Delo predstavi potrebno teoretično podlago za obstoj skritih markovskih modelov, opiše uporabo le-teh za tvorjenje besedil in ponazori njihovo uporabo v praksi.
V okviru diplomskega dela je opravljen tudi pregled obstoječih orodij za delo s skriti markovskimi modeli, medsebojna primerjava orodij in pregled njihove primernosti za uporabo pri tvorjenju besedil.
Opisan je postopek implementacije knjižnice za delo s skritimi markovskimi modeli in
težave, ki pri tem nastanejo.
Dve izmed pregledanih orodij in implementirana knjižnica so uporabljeni za tvorjenje besedil na podlagi korpusa slovenskega pisnega jezika.
Orodja so med seboj primerjana na podlagi tvorjenih besedil.
V diplomskem delu je opravljena tudi primerjava tvorjenih besedil s korpusom.

\bigskip

\noindent\textbf{Ključne besede:} \tkeywords.
% Empty page
\clearemptydoublepage

%%%%%%%%%%%%%%%%%%%%%%%%%%%%%%%%%%%%%%%%
% Abstract page
\selectlanguage{english}
\addcontentsline{toc}{chapter}{Abstract}
\chapter*{Abstract}

\noindent\textbf{Title:} \ttitleEn
\bigskip

\noindentNatural language generation (NLG) is the task of producing text that feels natural to the reader.
The goal of this diploma thesis is to study to which level natural language generation can be achieved using statistical models -- specifically hidden Markov models.
The diploma thesis covers probability and information theories that allow the definition of hidden Markov models and describes how such models can be used for the purpose of text generation.
Available tools for working with hidden markov models are reviewed, compared, and assesed for their suitability for generating text.
A library for hidden Markov models is implemented in Elixir.
Two of the reviewed tools and the implemented library are used to generate text from a corpus of written slovenian language.
A criterion for comparing generated texts is chosen and used to compare the models as well as comparing the generated texts to the corpus.

\bigskip

\noindent\textbf{Keywords:} \tkeywordsEn.
\selectlanguage{slovene}
% Empty page
\clearemptydoublepage

%%%%%%%%%%%%%%%%%%%%%%%%%%%%%%%%%%%%%%%%
\mainmatter
\setcounter{page}{1}
\pagestyle{fancy}

\include{chapters/intro}
\include{chapters/theory}
\include{chapters/model}
\include{chapters/compare}
\include{chapters/impl}
\include{chapters/bench}
\include{chapters/conclusion}

\addcontentsline{toc}{chapter}{Literatura}
\printbibliography
\label{ch:literatura}
\end{document}

