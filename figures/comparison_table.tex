\begin{table}
\begin{minipage}{\textwidth}
\small
\begin{tabular}{l lllll}
\toprule
                  & {\tt hmmlearn} & \tt{UMDHMM} & {\tt GHMM}& \tt{HMM} & \tt{mshmm} \\
\midrule
diskretni SMM     & da     & da      & da        & da        & da    \\
zvezni SMM        & da     & ne      & da        & da        & da    \\
mn. op. zap.$^a$  & ne$^b$ & da$^c$  & da        & da        & da    \\
razširljivost     & da     & ne      & ne        & da        & delna \\
knjižnjica        & da     & ne      & da        & da        & da    \\
dokumentacija     & da     & ne      & skopa     & ne        & skopa \\
licenca           & BSD    & GPL     & LGPL      & ni podana & GPL   \\
jezik             & Python & C       & C, Python & Python    & R     \\
\bottomrule
\end{tabular} \\
\vspace{6pt}{\footnotesize\indent \\
$\quad^a$ mnogotera opazovana zaporedja\\
$\quad^b$ Funkcija je sedaj že na voljo v novi različici knjižnjice. \\
$\quad^c$ Podpora za mnogotera opazovana zaporedja je omenjena v izvorni kodi, vendar uporabnik nima nadzora nad načinom, kako se vhod, ki je podan kot en niz, deli na več zaporedij.}
\caption{Primerjava funkcionalnosti in drugih lastnosti projektov.}
\label{tab:compare}
\end{minipage}
\end{table}
