\begin{table}
\begin{minipage}{\textwidth}
\centering
\small
\begin{tabular}{l|c|c|c|c|c|c}
& {\tt GHMM} & {\tt hmmlearn} & \tt{HMM} & \tt{mshmm} & \tt{UMDHMM} & \tt{Lawrence} \\ \hline
diskretno odd. & da & da & da & da & da & da \\ 
zvezno odd.    & da & da & da & da & ne & ne  \\
mn. op. zap.   & da & ne\footnote{Funkcija je sedaj že na voljo v novi različici knjižnjice.}
                        & da & da & da\footnote{Podpora za mnogotera opazovana zaporedja je omenjena v izvorni kodi, vendar uporabnik nima nadzora nad načinom, kako se vhod, ki je podan kot en niz, deli na več zaporedij.}
                                           & da \\
razširljivost  & ne & da & da & delna & ne & ne \\
knjižnjica  & da & da & da & da & ne & da \\
dokumentacija  & skopa & da & ne & skopa & ne & da \\
licenca        & LGPL & BSD & ni podana & GPL & GPL & MIT \\
jezik          & C, Python & Python & Python & R & C & Elixir \\
\end{tabular}
\caption[Pirmerjava funkcionalnosti projektov]{Primerjava funkcionalnosti in drugih lastnosti projektov, ki smo jih obravnavali. Primerjamo sposobnosti oddajanja diskretnih in zveznih vrednosti, podporo za mnogotera opazovana zaporedja, jezik v katerem je projekt napisan, kvaliteto dokumentacije in licenco, pod katero je izdana izvorna koda.}
\label{tab:compare}
\end{minipage}
\end{table}
